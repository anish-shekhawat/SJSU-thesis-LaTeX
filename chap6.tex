\chapter{Conclusion and Future Work\label{chap:conclusion}}

\section{Conclusion}

With an increase in worldwide adoption of HTTPS and advancement in malware detection techniques, we will see an increase in the number of malware samples using HTTPS and encryption to evade detection and hide their malicious activity. It is worrying because encryption interferes with the traditional detection techniques. Identifying such threats in a way that is feasible, fast and does not compromise user security is an important problem. Machine learning methods have been proven to overcome traditional limitations and can be used to train models on malware network traffic data. These models can then be used to detect similar malicious network traffic and flag a system for malware infection. Further, the system can be isolated to prevent further propagation of malware on the internal network.

The primary motivation of this research is the challenging problem of classifying encrypted network traffic as malicious or benign without using any decryption or deep packet inspection. In this research, we used several machine learning algorithms such as SVM, XGBoost, random forest to classify malicious and non-malicious encrypted network traffic. These algorithms were used to train and test models which can be used for classification. The results show that XGBoost performed better than other algorithms and reached the highest accuracy of $99.15$\%. We also achieved a high accuracy using only top six or top 10 features from Table \ref{tab:7}. Thus, we can conclude that encrypted malware network traffic is distinct from the normal traffic and can be used to successfully identify an infected host.

\section{Future Work}

There is a lot of scope to further improve the accuracy in future work. The next step would be to collect more data for training and testing the models and find additional features that could be useful in classification. Obtaining public network captures is harder due to the privacy issues involved. Thus, future work might include setting up a lab to generate both malicious as well as non-malicious network captures. The malicious network captures can be generated by running the latest malware samples that are uploaded to VirusTotal~\cite{VirusTotal} in a virtual environment and recording the network traffic. An interesting direction might be to try tools other than Bro to extract novel features and use them to try additional machine learning algorithms. Another research direction would be to classify the malicious traffic according to various malware families. Lastly, the future work might also include deploying the model on a real-world network to test the performance and robustness of the proposed approach.