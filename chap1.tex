\chapter{Introduction\label{chap:intro}}

%%%%% Cite a good reference (like Aycock's book) and give a standard definition of malware
Malware is, arguably, the greatest threat to information security today. Malware is defined as malicious software that can be used to gain access to a computer and steal, encrypt or destroy the data stored in the system. These malicious programs often communicate with a command and control (C\&C) server to fetch commands from an attacker. According to Malwarebytes, a popular cybersecurity product, over 90\%\ of small-to-medium sized businesses (SMB) have experienced an increase in the number of malware detection---some businesses experienced an increase of 500\%\  in
%%%%% Use "~" in place of a space, not in addition to a space
 March~2017 alone~\cite{Malwarebytes17}. Real-time malware detection using network traffic information has the potential to prevent---or at least greatly reduce---malware propagation on the network.

Commonly used malicious traffic detection techniques are based on deep packet inspection of the network traffic. 
%%%%% Need to cite a reference for statement such as that above---otherwise it sounds like your opinion
Such techniques aggregate successive packets that have the same protocol type and same source and destination port and address. They then analyze the contents of these aggregated packets and check for signatures that have been 
%%%%% Use "benign~\cite{SenSW04}" not "benign \cite{SenSW04}" or "benign ~\cite{SenSW04}."
previously discovered to classify the source as malicious or benign~\cite{SenSW04}.

%%%%% It's "HyperText", not "Hyper Text". Need to be consistent in usage...
Unfortunately, due to the widespread use of the HyperText Transfer Protocol Secure (HTTPS), or HTTP over Secure Socket Layer (SSL), such deep packet inspection methods can be inadequate to classify the network traffic. HTTPS is a secure communication protocol and, according to Google, more than 70\% of internet traffic is using HTTPS to communicate over the Internet~\cite{Google17}. 
%The report also states the desktop users using Chrome load more than 70\% of the websites 
%over HTTPS and Google receives only 5.9\% of desktop traffic which does not use HTTPS.

HTTPS is, essentially, HTTP using either the Secure Sockets Layer (SSL) or Transport Layer Security (TLS). Unencrypted traffic is exposed and can be read by anyone who can intercept the traffic packets. As everyday objects become more digital, many software applications and internet connected devices use encryption as their primary method to protect privacy, secure communications and maintain trust over the Internet. As a result, the rise in encrypted network traffic has affected the cybersecurity landscape. Malware can also leverage encryption by using it to evade detection and hide malicious activities. CISCO released a report~\cite{Anderson16} which states that although a majority of malware do not encrypt their network traffic, there is a steady 10\% to12\% increase annually in malicious network traffic that encrypt their communication using HTTPS. The~2017 Global Application \& Network Security Report from 
Radware~\cite{Radware17}, a leading cyber security and application delivery firm, states that 35\% of the organizations in their global security survey faced TLS or SSL based attacks, which represents an increase of 50\% over the previous year.

The main purpose of this research is to analyze whether we can effectively distinguish encrypted malicious network traffic from encrypted benign traffic, without decrypting. This paper is focused on the use of machine learning as a possible solution to this problem. Specifically, we use machine learning to analyze unencrypted network traffic based on a wide variety of unencrypted information, including connection duration, source and destination ports and IP addresses, SSL certificate key lengths, and so on.

The remainder of the paper is organized as follows. Chapter~\ref{chap:related} gives a basic overview of previous work related to the problem of detecting malicious traffic. In Chapter~\ref{chap:dataset}, we discuss the datasets used in our experiments. Chapter~\ref{chap:method}, discusses the proposed methodology. Chapter~\ref{chap:experiments} gives our experimental results and analysis. Finally, in Chapter~\ref{chap:conclusion}, we give our conclusion and briefly discuss future work.
