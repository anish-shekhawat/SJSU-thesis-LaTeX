\chapter{Introduction\label{chap:intro}}

The increase in popularity of smart devices has led to an increase in new vulnerabilities and sophisticated threats to information security. Malware has emerged to be the greatest threat to information security frameworks in 2016 . Malware is a malicious software program that can be used to gain access to a computer and steal, encrypt or destroy the data stored in the system. These malicious programs typically communicate with a Command and Control (C\&C) server to fetch commands from the attacker. According to Malwarebytes, a popular cybersecurity product, over 90 percent of Small-to-Medium sized businesses (SMB) have experienced an increase in the number of malware detections, whereas some businesses have even experienced an increase of 500 percent in March 2017 itself ~\cite{Malwarebytes17}. Real-time malware detection using network traffic information has the potential to prevent malware propagation on the network.

Commonly used malicious traffic detection techniques are based on deep packet inspection of the network traffic. They try to determine the source application of these communications by first aggregating successive packets that have the same protocol type and same source and destination port and address. They then analyze the contents of these aggregated packets and check for signatures that have been previously discovered to classify the source as malicious or benign \cite{SenSW04}.

Unfortunately, due to widespread adoption of Hyper Text Transfer Protocol Secure (HTTPS) or HTTP over Secure Socket Layer (SSL), such deep packet inspection methods are inadequate to classify the network traffic. HTTPS is a secure communication protocol used to communicate on the internet. According to Google, more than 70\% of the internet traffic is using HTTPS to communicate over the internet \cite{Google17}. The report also states the desktop users using Chrome load more than 70\% of the websites over HTTPS and Google receives only 5.9\% of desktop traffic which does not use HTTPS.

HTTPS is basically encrypted HTTP using either Secure Sockets Layer (SSL) or Transport Layer Security (TLS). An unencrypted traffic is exposed and can be read by anyone who can intercept the traffic packets. As everyday objects become more digital, a huge number of software applications and internet connected devices are using encryption as their primary method to protect privacy, secure communications and maintain trust over the internet. As a result, the rise in encrypted network traffic will also affect the cybersecurity landscape. Malware has started leveraging these benefits to their advantage by using them to evade detection and hide malicious activities. CISCO released a report \cite{Anderson16} which states that although a majority of malware do not encrypt their network traffic, there is a steady 10-12\% increase in malicious network traffic that encrypts their communication using HTTPS. The 2017 Global Application \& Network Security Report from Radware \cite{Radware17}, a leading cyber security and application delivery firm, states that 35\% of the organizations in their global security survey faced TLS or SSL based attacks which is an increase of 50\% over the previous year.

The main purpose of this paper is to analyze if an infected host can be identified by analyzing encrypted network traffic and without decrypting the traffic. This paper also evaluates the use of machine learning as a possible solution to identify infected host using unencrypted network traffic information such as connection duration, source and destination ports and IP addresses, SSL certificate key lengths, etc.

The rest of the paper is organized as follows: Chapter \ref{chap:related} gives a basic overview of related works used to detect malicious traffic. In Chapter \ref{chap:dataset} we look at the dataset used for the experiments. Chapter \ref{chap:experiments} explains the experiments performed and the results obtained and Chapter \ref{chap:conclusion} talks about the conclusion and future work.
